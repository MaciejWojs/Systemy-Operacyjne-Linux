\documentclass[a4paper]{article}
\usepackage{graphicx} % Required for inserting images
\usepackage{hyperref}
\usepackage[polish]{babel}
\usepackage{polski}
\usepackage{array}
\usepackage{biblatex}
\addbibresource{bibliography.bib}
\usepackage{csquotes}
\usepackage[shortlabels]{enumitem}
%BinaryBuilders 

\newcommand{\smaller}[1]{{\large #1}}

\newcommand*{\fg}[4][!htb]{
      \begin{figure*}[#1]
      \includegraphics[#2]{#3}
      \caption{#4}
      \end{figure*}
}

\title{\textbf{Akademia Nauk Stosowanych w Nowym Sączu}\\\smaller{Wydział Nauk Inżynieryjnych}\vspace*{1em} \\ \textbf{Systemy operacyjne – projekt}\vspace*{0.5em}\\\smaller{studia stacjonarne\\semestr letni 2023/2024}}
\author{}
\date{}

\begin{document}
\maketitle

\textbf{Temat projektu:}
\begin{enumerate}
      \item Zaprojektować infrastrukturę informatyczną na potrzeby firmy BinaryBuilders. Realizacja
      serwerowa w oparciu o system operacyjny Linux, np.\@ Fedora Server 39, stacje klienckie
      np.\@ Linux MINT. %chktex 13

    \item Wdrożyć niezbędne usługi wynikające z założeń takie jak: SSH, DHCP, DNS, HTTP/S,
          motor bazodanowy (MySql)+PHP+phpMyAdmin, CMS WordPress, RAID, SAMBA,
          SQUID, Postfix(SMTP) + Dovecot(POP/IMAP), oraz wybraną usługę. Wdrożyć %chktex 36
          automatyzację przy użyciu skryptu np. Bash, oraz usługi cron.
    \item Cele projektu zweryfikować z założeniami zapisanymi w dokumencie „Szczegółowy
          zarys projektu”.
\end{enumerate}

% Increase row height by 50%
\renewcommand{\arraystretch}{1.25}
% Increase column spacing
\setlength{\tabcolsep}{12pt}

\begin{table}[h!]
    \centering
    \begin{tabular}{cccccc}
    Imie i nazwisko: &&&& Data oddania: \\
    Maciej Wójs &&&& \today \\
    Nr grupy: &&&& Ocena: \\
    L3 &  \\
    \end{tabular}
    % \caption{Sample Table}
    % \label{tab:sample_table}
\end{table}

\newpage
\tableofcontents
\newpage
\listoffigures
\newpage

\section{Założenia projektowe – wymagania}
\begin{enumerate}[a)] % chktex 9  chktex 10
      \item Systemy operacyjne: Fedora Server 39 lub inny serwer z rodziny Linux, oraz system kliencki 
      np. Linux MINT. 
      \item zarządzanie serwerem poprzez SSH, oraz emulator putty.exe 
      \item nazwa serwera ma być zgodna z nazewnictwem: svrXX-firma, gdzie XX oznaczają dwie 
      ostatnie cyfry numeru albumu wykonawcy, a firma to skrót nazwy swojej 
      firmy (niepowtarzalny) – wymyślonej, 
      \item na podstawie nazwy firmy należy założyć lokalną domenę o nazwie np. firma.ns i 
      skonfigurować usługę DNS Server, 
      \item adres IP serwera, zakres adresacji IP, oraz brama domyślna od strony sieci wewnętrznej 
      VirtualBOXa (sieć LAN firmy) w której ma działać serwer DHCP ma mieć następujące 
      wartości:
      \vspace{-2mm}
      % \begin{table}[h]
            \begin{center}
                  \begin{tabular}{cc}
                        adres IP:& 192.168.230.1/24,\\
                        zakres:& 192.168.230.10–60\\
                        brama domyślna:  & 192.168.230.1\\
                  \end{tabular}
            \end{center}
      % \end{table}
      \vspace{-2mm}
      \item należy utworzyć macierz dyskowa programową na poziomie RAID 5 z dyskiem zapasowym. 
      Uzyskać wypadkową pojemności macierzy 10GB. Przestrzeń macierzy podzielić na dwie 
      równe partycje, 
      \item Pierwszą partycję zamontować do punktu \textbf{/dysksieciowy}, a drugą do punktu \textbf{/kopie}. 
      Zapewnić ich automatyczne montowanie podczas startu systemu, 
      \item  serwer ma udostępniać zasób sieciowy o adresie UNC \textbf{\textbackslash \textbackslash sfs.firma.ns\textbackslash dysk} odnoszący się do systemu plików \textbf{/dysksieciowy} (ppkt. g), 
      \item należy wdrożyć usługę WEB Server z obsługą PHP, oraz serwer bazodanowy zarządzany przez 
      phpMyAdmin, oraz CMS WordPress, skonfigurować UserDir dla WEB Serwer'a,
      \item dostęp do sieci Internet z sieci wewnętrznej ma się odbywać za pośrednictwem serwera 
      PROXY(squid), a aktywność pracowników firmy ma być monitorowana,
      \item w firmie należy wdrożyć serwer pocztowy, oraz klienta mail, 
      \item zapewnić aby popularne usługi były dostępne jako oddzielne nazwy hostów, jak np.: 
            \begin{itemize}
                  \item \textbf{www.firma.ns} (serwer www), 
                  \item \textbf{poczta.firma.ns} (serwer poczty), 
                  \item \textbf{sfs.firma.ns} (serwer samby), 
            \end{itemize}
      \item wdrożyć automatyczną archiwizację systemu plików /home zawierającego katalogi użytkowników. Archiwizacja ma rozpoczynać się automatycznie codziennie o 21:00. W wyniku archiwizacji ma powstać plik\\ \textbf{home\_20240510.tar.gz} zapisany w \textbf{/kopie} (ppkt. g)
      \item Dodatkowo wdrożyć dowolną usługę, ale taką która nie była wdrażana podczas zajęć.
       
\end{enumerate}
% \fg{contents/configuration/SSH/1.png}
\section{Opis użytych technologi}
\subsection{SSH (Secure Shell)}
SSH to protokół sieciowy, który umożliwia bezpieczne zdalne logowanie oraz wykonywanie poleceń na odległym serwerze. Zapewnia szyfrowanie komunikacji, co chroni przed podsłuchiwaniem oraz atakami typu man-in-the-middle.

\subsection{DHCP (Dynamic Host Configuration Protocol)}
DHCP to protokół używany do automatycznego przydzielania adresów IP i innych parametrów konfiguracyjnych urządzeniom w sieci. Ułatwia zarządzanie siecią poprzez automatyczne przypisywanie ustawień.

\subsection{DNS (Domain Name System)}
DNS to system, który przekształca łatwe do zapamiętania nazwy domen (np.\ www.example.com) na adresy IP, które są wykorzystywane przez urządzenia sieciowe do komunikacji. DNS działa jak książka telefoniczna internetu.

\subsection{HTTP/S (Hypertext Transfer Protocol/Secure)}
HTTP to protokół komunikacyjny używany do przesyłania stron internetowych. HTTPS to jego bezpieczna wersja, która wykorzystuje TLS/SSL do szyfrowania danych, zapewniając poufność i integralność komunikacji między przeglądarką a serwerem.

\subsection{MySQL}
Popularny system zarządzania relacyjnymi bazami danych. Umożliwia przechowywanie i zarządzanie dużą ilością danych w strukturach tabelarycznych.

\subsection{PHP}
Skryptowy język programowania, często używany do tworzenia dynamicznych stron internetowych. PHP może komunikować się z bazami danych, takimi jak MySQL.

\subsection{phpMyAdmin}
Narzędzie webowe do zarządzania bazami danych MySQL. 
Umożliwia wykonywanie operacji na bazach danych za pomocą interfejsu graficznego.

\subsection{CMS WordPress}
WordPress to system zarządzania treścią (CMS), który pozwala na łatwe tworzenie i zarządzanie stronami internetowymi. Jest bardzo popularny ze względu na swoją elastyczność, prostotę obsługi oraz bogaty ekosystem wtyczek i motywów.

\subsection{RAID (Redundant Array of Independent Disks)}
RAID to technologia, która łączy kilka dysków twardych w jedną jednostkę logiczną w celu poprawy wydajności i/lub redundancji danych. Istnieje kilka poziomów RAID, z których każdy oferuje różne kombinacje wydajności i bezpieczeństwa danych.

\subsection{SAMBA}
SAMBA to pakiet oprogramowania, który umożliwia integrację systemów operacyjnych Linux/Unix z sieciami Windows. Pozwala na udostępnianie plików i drukarek w sieci oraz współpracę z domenami Windows (Active Directory).

\subsection{SQUID}
SQUID to serwer proxy i buforujący, który może przyspieszyć dostęp do zasobów internetowych poprzez przechowywanie często używanych danych w lokalnej pamięci podręcznej. Może również służyć jako filtr treści i narzędzie do monitorowania ruchu sieciowego.

\subsection{Postfix (SMTP) + Dovecot (POP/IMAP)}
\subsubsection{Postfix}
Serwer pocztowy obsługujący protokół SMTP, używany do wysyłania i odbierania wiadomości e-mail. Jest znany z wydajności i bezpieczeństwa.

\subsubsection{Dovecot}
Serwer IMAP i POP3 używany do odbierania i przechowywania wiadomości e-mail. Jest zoptymalizowany pod kątem wydajności i bezpieczeństwa, oferując wsparcie dla nowoczesnych standardów pocztowych.

\subsection{Automatyzacja za pomocą skryptów Bash i usług cron}
\subsubsection{Skrypty Bash}
Skrypty napisane w Bash (Bourne Again Shell) służą do automatyzacji zadań w systemach Unix/Linux. Mogą być używane do instalacji oprogramowania, konfiguracji systemu, zarządzania plikami i wielu innych zadań.

\subsubsection{cron}
Usługa systemowa w Unix/Linux, która pozwala na planowanie zadań do wykonania w określonym czasie lub regularnych odstępach czasu. Jest używana do automatyzacji zadań takich jak backup, aktualizacje systemu czy uruchamianie skryptów.

\section{Schemat logiczny projektowanej infrastruktury sieciowej}
\fg{width=\textwidth}{contents/VirtualBox-network-setup/setup5.png}{Schemat logiczny sieci}
CCCCCCCCCC
Test TAG
\fg{width=\textwidth}{contents/configuration/SSH/1.png}{ssh}
\fg{width=\textwidth}{contents/configuration/SSH/2.png}{podlaczenie poprzez putty}
\section{Procedury instalacyjne poszczególnych usług}
DDDDDDDDDDDDDD
\section{Testy działania wdrożonych usług}
EEEEEEEEEEE
\section{Kod skryptu BASH, oraz tablica crontab}
FFFFFFFFFFFFFFF
\section{Wnioski}
EEEEEEEEEEEEEE
\newpage
\nocite{k8s-docs}
\nocite{k8s-blog}
\nocite{k8s-github}
\printbibliography[heading=bibnumbered, label=Literatura, title=Literatura]
\end{document}
