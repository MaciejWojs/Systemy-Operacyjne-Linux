\documentclass[a4paper]{article}
\usepackage{graphicx} % Required for inserting images
\usepackage{hyperref}
\usepackage[polish]{babel}
\usepackage{polski}
\usepackage{array}
\usepackage{biblatex}
\addbibresource{bibliography.bib}

\title{\textbf{Akademia Nauk Stosowanych w Nowym Sączu\\Wydział Nauk Inżynieryjnych\newline \newline Systemy operacyjne – projekt\\studia stacjonarne\\semestr letni 2023/2024}}
% \author{Imie nazwisko}
\date{}

\begin{document}
\maketitle

\textbf{Temat projektu:}
\begin{enumerate}
    \item Zaprojektować infrastrukturę informatyczną na potrzeby „wybranej firmy”. Realizacja
    serwerowa w oparciu o system operacyjny Linux, np.: Fedora Server 39, stacje klienckie
    np. Linux MINT.
    \item Wdrożyć niezbędne usługi wynikające z założeń takie jak: SSH, DHCP, DNS, HTTP/S,
    motor bazodanowy (MySql)+PHP+phpMyAdmin, CMS WordPress, RAID, SAMBA,
    SQUID, Postfix(SMTP) + Dovecot(POP/IMAP), oraz wybraną usługę. Wdrożyć
    automatyzację przy użyciu skryptu np. Bash, oraz usługi cron. 
    \item Cele projektu zweryfikować z założeniami zapisanymi w dokumencie „Szczegółowy
    zarys projektu”. 
\end{enumerate}

% Increase row height by 50%
\renewcommand{\arraystretch}{1.25}

% Increase column spacing
\setlength{\tabcolsep}{12pt}

\begin{table}[h!]
    \centering
    \begin{tabular}{cccccc}
    Imie i nazwisko: &&&& Data oddania: \\
    Maciej Wójs &&&& \today \\
    Nr grupy: &&&& Ocena: \\
    L3 &  \\
    \end{tabular}
    % \caption{Sample Table}
    % \label{tab:sample_table}
\end{table}

\newpage
\tableofcontents
\newpage

\section{Założenia projektowe – wymagania}
Test TAG 
\section{Opis użytych technologi}
BBBBBBBBBBBB
\section{Schemat logiczny projektowanej infrastruktury sieciowej}
CCCCCCCCCC
\section{Procedury instalacyjne poszczególnych usług}
DDDDDDDDDDDDDD
\section{Testy działania wdrożonych usług}
EEEEEEEEEEE
\section{Kod skryptu BASH, oraz tablica crontab}
FFFFFFFFFFFFFFF
\section{Wnioski}
EEEEEEEEEEEEEE
\newpage
\nocite{k8s-docs}
\nocite{k8s-blog}
\nocite{k8s-github}
\printbibliography[heading=bibnumbered, label=Literatura, title=Literatura]
\end{document}
